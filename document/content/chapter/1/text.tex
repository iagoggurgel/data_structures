%%%%%%%%%%%%%%%%%%%%%%%%%%%%%%%%%%%%%%%%%%%%%%%%%%%%%%%%%%%%%%%%%%%%%%%%%%%%%
%% INÍCIO CAPÍTULO                                                         %%
%%%%%%%%%%%%%%%%%%%%%%%%%%%%%%%%%%%%%%%%%%%%%%%%%%%%%%%%%%%%%%%%%%%%%%%%%%%%%

\chapter{Introdução}
\label{cap:1:intro}

\citacao{%
He who knows all the answers has \\ not been asked all the questions.}
{Confucius}


O estudo e análise de algoritmos é fundamental para o desenvolvimento do conhecimento
no estudo da Computação. Para isso, foi realizado um estudo baseado nos conteúdos
da disciplina de Estrutura de Dados ministrada no curso de Bacharelado de Sistemas
de Informação da UFRN com o material adicional \cite{cormen2022algorithms} com objetivo de
compreender profundamente os conteúdos da disciplina.

\section{O que são algoritmos e qual a sua importância no mundo real?}

Para \cite{cormen2022algorithms}, os algoritmos são qualquer procedimento computacional bem
definido capaz de produzir um conjunto de valores como saída, a partir de um conjunto de valores
como entrada. Um algoritmo também pode ser descrito como um conjunto de instruções ou passos que
devem ser executados com objetivo de produção de um valor significativo para o contexto qual foi
executado. Sua importância é fundamental para qualquer sistema computacional, como exemplo, pode-se
imaginar o problema de ordenação de valores, sendo este, um exemplo bastante recorrente na computação,
o mesmo pode ser definido de maneira formal como: \\

\textbf{Input: } Uma sequência de n números $(a_1, a_2, a_3,\ldots, a_n)$ \\

\textbf{Output: } Uma reordenação $(a_1, a_2, a_3,\ldots, a_n)$ da sequência para qual \\
$a_1 \leq a_2 \leq a_3 \leq \cdots \leq a_n$. \\

Para medir a importância dos algoritmos na sociedade, podemos visualizar algumas das suas aplicações no mundo
real:

\begin{enumerate}
    \item Ordenação de valores
    \item Compressão de dados
    \item O menor caminho entre dois locais
    \item Cálculos estatísticos
    \item Segurança
    \item Transformações em dados
\end{enumerate}

Para cada uma dessas aplicações, existem algoritmos que solucionam o problema em questão. No entanto, é importante
ressaltar que, entre as múltiplas implementações que solucionam um problema, nem todas serão implementações ótimas,
ou seja, otimizadas e eficientes para solucionar o problema no menor tempo possível. No capítulo \ref{cap:2:sorting},
será possível observar as diferenças em tempo de execução de diferentes implementações de soluções para um mesmo
problema.

\section{O que são estruturas de dados?}

As estruturas de dados são, de acordo com \cite{cormen2022algorithms}, formatos de armazenamento de dados
que permitem fácil acesso e modificação. Também, permitem organizar os dados de maneiras diferentes, possibilitando assim que
algoritmos manipulem os dados de maneiras diferentes. \cite{narasimha2017data} discorre sobre a divisão das estruturas
de dados em 2 tipos:

\begin{itemize}
    \item Estruturas de dados lineares: Onde os elementos são acessados de forma sequencial. Exemplos como: Listas encadeadas, Pilhas e Filas.
    \item Estruturas de dados não-lineares: Onde os elementos são armazenados e acessados de forma não-linear. Exemplos: Árvores e grafos.
\end{itemize}

\section{Procedimento de análise de eficiência de um algoritmo}

Para realizar a análise e medição de um algoritmo, é necessário compreender o algoritmo e seu contexto. Para isso, vamos
tomar como exemplo o algoritmo de busca por um elemento em um vetor de forma linear. Esse algoritmo é nomeado de busca
linear e pode ser visto em \ref{linearSearch}.

\begin{pseudocode}[caption={Algoritmo de Busca linear}, label={linearSearch}]
LINEAR-SEARCH(A, x)
n $\gets$ len(A)
for y from 1 to n
    if A[y] = x then
        return y
return NILL
\end{pseudocode}

Ao analisar o algoritmo apresentado, pode-se compreender que seu funcionamento é baseado em percorrer
de forma sequencial e linear o vetor testando pelo elemento a ser buscado. Se esse elemento existir no vetor,
seu índice será retornado como valor da função, se não, será retornado o valor nulo. Para medir sua eficiência,
é necessário calcular seu tempo de execução. Para isso, é necessário perceber que a função de busca linear é
intrinsicamente associada ao tamanho $n$ do vetor $A$ e a probabilidade da existência de $x$ no vetor. Imaginando que
cada linha do pseudo-código apresentado em \ref{linearSearch} executa em um tempo específico, podemos pensar a
equação \ref{cap:1:eq:linearSearch} para seu tempo de execução de pior caso. Observe que: para a linha z do código
temos $C_z$ como seu tempo de execução.

\begin{equation} \label{cap:1:eq:linearSearch}
    T(n) = C_2 + \sum_{i=1}^{n}(C_3 + C_4) + C_5 + C_6
\end{equation}

Como definido em \cite{cormen2022algorithms}, pode-se expressar essa equação como $T(n) = an + b$, com $a = (C_3 + C_4)$
e $b = (C_2 + C_5 + C_6)$ mas, antes que seja determinado a ordem do tempo de execução dessa função, é fundamental compreender
a segunda determinante para sua métrica de eficiência. Um fator previamente mencionado foi que, para realizar-se a medida do tempo
de execução, também é necessário compreender a probabilidade $p$ que um elemento qualquer $x$ esteja presente nesse vetor $A$, como
apresenta a equação \ref{cap:1:eq:probabilityProof}.

\begin{equation} \label{cap:1:eq:probabilityProof}
    p = \frac{n}{\mathbb{Z}}
\end{equation}

Ajustando a equação $T(n) = an + b$ com a probabilidade $p$ de que $x$ pertença a $A$, percebe-se que:

\begin{equation*} 
    T(n) = (1 - p)(an + b) + p(an + b)
\end{equation*}

Com isso, é possível determinar que o tempo de execução do algoritmo é linear, já que a probabilidade $p$ de que
$x$ pertença a $A$ é muito baixa, e mesmo que $x \in A$, a probabilidade de ocorrência do pior caso é mais alta
que do melhor caso, sendo o pior caso $T(n) = an + b$ e o melhor caso $T(n) = 1$.

\section{Notação assintótica e definições formais}

Ao realizar a análise de um algoritmo, é possível perceber a sua razão de crescimento. O estudo das razões de
crescimento do tempo de execução dos algoritmos é o estudo da eficiência assintótica de um algoritmo, isso é,
qual a razão de um crescimento de um algoritmo qualquer quando o número de entradas do mesmo tende ao infinito.

\subsection{Notação $O(n)$}

A notação $O(n)$ indica o limite superior da razão de crescimento em uma função de comportamento assintótico.
Para o caso do algoritmo \ref{linearSearch}, pode-se determinar que:

\begin{equation*} 
    O(n) = n^x \:  \forall \:  x \geq 1
\end{equation*}

Ou também, qualquer função com razão de crescimento superior ao cresimento linear pode ser usado como $O(n)$ da
função de busca linear.

\subsection{Notação $\Omega(n)$}

A notação $\Omega(n)$ indica o limite inferior da razão de crescimento em uma função de comportamento assintótico.
Para o caso do algoritmo \ref{linearSearch}, pode-se determinar que:

\begin{equation*} 
    \Omega(n) = n^x \:  \forall \:  x \leq 1
\end{equation*}

Ou também, qualquer função com razão de crescimento inferior ao cresimento linear pode ser usado como $\Omega(n)$ da
função de busca linear.

\subsection{Notação $\Theta(n)$}

A notação $\Theta(n)$ indica o limite justo da razão de crescimento em uma função de comportamento assintótico.
Determina-se $\Theta(n)$ de uma função qualquer provando que seu cresimento é tanto $O(f(n))$ como
$\Omega(f(n))$, dessa forma, prova-se que essa função é $\Theta(f(n))$.
Para o caso do algoritmo \ref{linearSearch}, pode-se determinar que:

\begin{align*}
    O(n) &= n \\
    \Omega(n) &= n \\
    \Theta(n) &= n
\end{align*}

Dessa forma, é possível determinar que $\Theta(n) = n$ para a busca linear.


%%%%%%%%%%%%%%%%%%%%%%%%%%%%%%%%%%%%%%%%%%%%%%%%%%%%%%%%%%%%%%%%%%%%%%%%%%%%%
%% FIM CAPÍTULO                                                            %%
%%%%%%%%%%%%%%%%%%%%%%%%%%%%%%%%%%%%%%%%%%%%%%%%%%%%%%%%%%%%%%%%%%%%%%%%%%%%%
