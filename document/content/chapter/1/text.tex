%%%%%%%%%%%%%%%%%%%%%%%%%%%%%%%%%%%%%%%%%%%%%%%%%%%%%%%%%%%%%%%%%%%%%%%%%%%%%
%% INÍCIO CAPÍTULO                                                         %%
%%%%%%%%%%%%%%%%%%%%%%%%%%%%%%%%%%%%%%%%%%%%%%%%%%%%%%%%%%%%%%%%%%%%%%%%%%%%%

\chapter{Introdução}
\label{cap:1:intro}

\citacao{%
If knowledge can create problems,\\
it is not through ignorance that we can solve them.}
{Isaac Asimov}

O estudo e análise de algoritmos é fundamental para o desenvolvimento do conhecimento
no estudo da Computação. Para isso, foi realizado um estudo baseado nos conteúdos
da disciplina de Estrutura de Dados ministrada no curso de Bacharelado de Sistemas
de Informação da UFRN com o material adicional \cite{cormen2022algorithms} com objetivo de
compreender profundamente os conteúdos da disciplina.

\section{O que são algoritmos e qual a sua importância no mundo real?}

Para \cite{cormen2022algorithms}, os algoritmos são qualquer procedimento computacional bem
definido capaz de produzir um conjunto de valores como saída, a partir de um conjunto de valores
como entrada. Um algoritmo também pode ser descrito como um conjunto de instruções ou passos que
devem ser executados com objetivo de produção de um valor significativo para o contexto qual foi
executado. Sua importância é fundamental para qualquer sistema computacional, como exemplo, pode-se
imaginar o problema de ordenação de valores, sendo este, um exemplo bastante recorrente na computação,
o mesmo pode ser definido de maneira formal como: \\

\textbf{Input: } Uma sequência de n números $(a_1, a_2, a_3,\ldots, a_n)$ \\

\textbf{Output: } Uma reordenação $(a_1, a_2, a_3,\ldots, a_n)$ da sequência para qual \\
$a_1 \leq a_2 \leq a_3 \leq \cdots \leq a_n$. \\

Para medir sua importância, podemos visualizar algumas das aplicações dos algoritmos no mundo
real:

\begin{enumerate}
    \item Ordenação de valores
    \item Compressão de dados
    \item O menor caminho entre dois locais
    \item Cálculos estatísticos
    \item Segurança
    \item Transformações em dados
\end{enumerate}

Para cada uma dessas aplicações, existem algoritmos que solucionam o problema em questão. No entanto, é importante
ressaltar que, entre as múltiplas implementações que solucionam um problema, nem todas serão implementações ótimas,
ou seja, otimizadas e eficientes para solucionar o problema no menor tempo possível. No capítulo \ref{cap:2:sorting},
será possível observar as diferenças em tempo de execução de diferentes implementações de soluções para um mesmo
problema.

\section{O que são estruturas de dados?}

As estruturas de dados são, de acordo com \cite{cormen2022algorithms}, formatos de armazenamento de dados
que permitem fácil acesso e modificação.

\section{Procedimento de análise de eficiência de um algoritmo}

%%%%%%%%%%%%%%%%%%%%%%%%%%%%%%%%%%%%%%%%%%%%%%%%%%%%%%%%%%%%%%%%%%%%%%%%%%%%%
%% FIM CAPÍTULO                                                            %%
%%%%%%%%%%%%%%%%%%%%%%%%%%%%%%%%%%%%%%%%%%%%%%%%%%%%%%%%%%%%%%%%%%%%%%%%%%%%%
